\chapter{objects.\+txt syntax}
\hypertarget{md__c_1_2_users_2namph_2_downloads_2openssl_2openssl-3_82_81_2crypto_2objects_2_r_e_a_d_m_e}{}\label{md__c_1_2_users_2namph_2_downloads_2openssl_2openssl-3_82_81_2crypto_2objects_2_r_e_a_d_m_e}\index{objects.txt syntax@{objects.txt syntax}}
To cover all the naming hacks that were previously in {\ttfamily objects.\+h} needed some kind of hacks in {\ttfamily objects.\+txt}.

The basic syntax for adding an object is as follows\+: \begin{DoxyVerb}    1 2 3 4         : shortName     : Long Name

            If Long Name contains only word characters and hyphen-minus
            (0x2D) or full stop (0x2E) then Long Name is used as basis
            for the base name in C. Otherwise, the shortName is used.

            The base name (let's call it 'base') will then be used to
            create the C macros SN_base, LN_base, NID_base and OBJ_base.

            Note that if the base name contains spaces, dashes or periods,
            those will be converted to underscore.
\end{DoxyVerb}
 Then there are some extra commands\+: \begin{DoxyVerb}    !Alias foo 1 2 3 4

            This just makes a name foo for an OID.  The C macro
            OBJ_foo will be created as a result.

    !Cname foo

            This makes sure that the name foo will be used as base name
            in C.

    !module foo
    1 2 3 4         : shortName     : Long Name
    !global

            The !module command was meant to define a kind of modularity.
            What it does is to make sure the module name is prepended
            to the base name.  !global turns this off.  This construction
            is not recursive.
\end{DoxyVerb}
 Lines starting with {\ttfamily \#} are treated as comments, as well as any line starting with ! and not matching the commands above. 